\documentclass[main.tex]{subfiles}

\begin{document}

This text is second edition of the revised lab manual for PHYS-1154 a one-semester laboratory course for students enrolled in PHYS 1110 or 2110. This manual covers experiments in mechanics, pressure, thermodynamics and sound. I have endeavored to write this book so that it is accessible to students with only knowledge in algebra while including several calculus concepts so that we can actually talk about the physics that we are studying. The first revised edition of this manual was released in August 2014 and sought to update the manual written by Dr. Glenn Sowell in 2003. This mainly consisted of moving the text into an unified format across the book as well as updating the procedures to reference instructions in PASCO's Capstone software.

Changes in this edition include many, many small fixes across the manuscript that both my colleagues and my students pointed out to improve the manuscript. Also, some larger structural changes to the course:  The section on vector arithmetic has been split from kinetic friction in the fourth chapter and moved to the first chapter where it is now paired with a section on general measurements and uncertainty. This will hopefully adjust the timing of the course to match closer with the lecture course and help reinforce the concepts learned in both courses. To compensate for the loss of the vectors section, the friction chapter now includes a section on static friction. Archimedes' principle has been moved to the sixth chapter since it utilizes primarily forces to determine the density of our objects. Also, the monster chapter of torque and angular momentum has finally been split! This is something that I have wanted to do ever since I took the class myself. Torque now has its own dedicated chapter and now includes a section on dynamic torque where students will explore the relation between torque, moment of inertia, and angular acceleration. I have left angular momentum largely unchanged as I felt that the section and experiment was more than enough all by itself.  The harmonics and calorimetry chapters have swapped places so that they follow the progression found in most lecture textbooks. Finally, the standing waves chapter that used to end our semester has been dropped from the course. None of the lecture courses ever got to the chapters on standing waves and with the splitting of torque and angular momentum an extra week was needed so unfortunately standing waves lost to scheduling.

\begin{flushright}
Benjamin Floyd
\end{flushright}•

\end{document}