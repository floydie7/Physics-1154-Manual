\documentclass[main.tex]{subfiles}

\begin{document}

\emph{Any new instructions given by the instructor will, of course, supersede those given here.}

It is important to note that the report is to be an individual project even though the data was taken as a group.  This does not mean that other members of the group cannot be contacted and asked questions but the report should be in the words of its author.  If texts or other written material is used, any material from such sources utilized in the report should be cited in the report by either using footnotes or a bibliography. Data sheets must accompany the report so that the raw data is represented unaltered.  Photocopied work will be considered as representing academic dishonesty as will any work that appears to be too close to that of another person.

The report should be clear, concise, and complete (self-explanatory). It should not contain references to this manual. When deriving equations, explain what is being done in both words and mathematical notation. For examples of this, look at the theory sections in this manual. Remember when doing calculations that a numerical answer without units is the same as having no answer at all. If an answer is truly unitless such as a ratio of the same quantity make sure to indicate that there are no units. If there are a large number of similar calculations, show one sample then place the results in a table. In addition to the general guidelines for a minor report, a major report should be more detailed in all sections. A report should include the following sections:
\begin{enumerate}
\item
Objective:\\
State what the experiment is attempting to do. Is the object to measure a given quantity, to verify a given number such as $g\;(=9.8\text{m}/\text{s}^2),$ to verify a concept, or to determine the relationship between several variables? It might be a combination of these.
\item
Theory:\\
All formulas used in the experiment are to be derived or given as definitions. For example, in the centripetal force experiment the formula $a=v^2/r$ would be derived as would the other equations used. However a formula such as $\rho=m/V$ is a defining equation for the density and should be given as such.

In deriving equations, descriptive wording should be used to tell what is being done. For example, do not say ``$v=2\pi r/t,$" instead say ``The speed is given by the distance around a circular path divided by the time it takes to complete the circle. Instead of dividing by the time for one revolution, we can multiply $2\pi r$ by $n,$ the number of revolutions per unit time. Then $v=2\pi nr.$" Also discuss the concepts fully, explain what the terms mean. e.g., explain what centripetal force, torque, buoyant force, etc. mean.
\item
Method and equipment (\emph{Major Report}):\\
State the general method to be used to meet the objective stated above. An example might be: ``Numerous measurements of the length and width of a metal plate will be used to study the relation between the magnitude of both quantities, the magnitude of the uncertainties, and the relative uncertainties. Measurements of the same rod with three different instruments having inherently different precision will be used to determine the relation between the precision of the measurement and the relative uncertainty."

This section also includes a list of equipment used to preform the experiment.
\item
Procedure:\\
Briefly outline everything that was done to preform the experiment. This should be in the words of the author only.
\item
Data Processing:\\
A table of all raw measurements should be shown here. Show how all calculations were done in this section. Include any graphs asked for as well as any interpretations requested. A sample of each different calculation including both numbers and units must be included. Generally, only one sample of each type of calculation is needed. If there is any difference, other than numerical values, then a new sample should be shown.
\item
Error Analysis (\emph{Major Report}):
Discuss the reason for any large relative uncertainties. If the results of two or more calculations are to be compared and the results do not overlap, determine either the relative discrepancy or the percent discrepancy as explained in the Uncertainty section on page~\pageref{sec:Uncertainty}. Also discuss the reasons for discrepancies, especially any large discrepancies.
\item
Questions and Problems:\\
Any questions or problems stated in the instructions of the experiment should be answered.
\item
Conclusions:\\
What relationships between quantities did the experiment reveal? What physics principles or concepts were verified? Any suggestions for improving the experiment as a learning experience would be appreciated and should be given here.
\end{enumerate}•

\end{document}
