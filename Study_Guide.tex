\documentclass[main.tex]{subfiles}

\begin{document}

At the beginning of your lab period, on the scheduled test day, go to the testing room in the Durham Computer Lab, DSC 114. (The testing room in 104 is the enclosed room of computers with the glass windows.) Hand in any work to your lab instructor, pick up graded papers, and sit down at a computer. Wait for further instructions from your lab instructor.

The test will cover all labs! You need to understand the physics of each of these labs and know how to reproduce the calculations from them. You need to understand how the various sensors work. The test will be mostly multiple choice, true/false and short answer. They may be one or two longer answers required.

Listed below are a few sample questions that you should know how to answer.

\begin{enumerate}
\item
In several of the labs you had to calibrate a Force probe. Describe the process in detail.
\item
A student doing the lab in which the acceleration of a freely-falling picket fence was determined, forgot to label one of her graphs. How should the vertical axis be labeled?
\item
In the centripetal force lab, describe in detail the procedure for determining the centripetal force.
\item
Describe an experiment to verify the conservation of mechanical energy. What analysis must be done to verify the conservation law?
\item
A student measures the acceleration due to gravity to be $9.2 \text{m}/\text{s}^2, 8.9 \text{m}/\text{s}^2,$ $9.1 \text{m}/\text{s}^2, 9.5 \text{m}/\text{s}^2.$ What is the mean and standard deviation for his experiment? Is there evidence of a random error? A systematic error?
\item
If you use a Photogate to measure the acceleration of a cart, then what should you use on top of the cart? A single block or multiple blocks? Why?
\item
In the Archimedes experiment, why is it necessary to tie a heavy ball onto the light ball in order to determine the density of the light ball?
\item
One way we determined $g,$ the acceleration due to gravity, was to measure the acceleration of a cart down an incline. Describe one way of measuring the
acceleration of the cart. What is the formula relating acceleration $a,$ $g,$ and the angle of the incline?
\item
If you are comparing your experimental results to a ``standard'' result, what type of error should you use?
\item
(T/F) The Force sensor is designed so that pulling on the hook produces a negative force.
\item
Describe the procedure for using Capstone to display a graph of ``Force vs. Acceleration." It is not necessary to describe the calibration of the Force Sensor.
\item
In the Sound Lab, how is the distance between two antinodes related to the wavelength of the sound? Is it necessary to know the speed of sound for this?
\item
(T/F) If the data for a ``Position vs. Time" graph is smooth, then the corresponding ``Acceleration vs. Time" graph must also be smooth. Explain.
\item
How is a Photogate Pulley similar to a Photogate? How is it different?
\end{enumerate}•

\end{document}
